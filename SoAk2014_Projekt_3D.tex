\documentclass[
  bibliography=totoc,     % Literatur im Inhaltsverzeichnis
  captions=tableheading,  % Tabellenüberschriften
  titlepage=firstiscover, % Titelseite ist Deckblatt
]{scrartcl}

\usepackage{fixltx2e}
\usepackage[aux]{rerunfilecheck}

\usepackage{polyglossia}
\setmainlanguage{german}

\usepackage{amsmath}
\usepackage{amssymb}
\usepackage{mathtools}

\usepackage{fontspec}
\defaultfontfeatures{Ligatures=TeX}

\usepackage[
  math-style=ISO,
  bold-style=ISO,
  sans-style=italic,
  nabla=upright,
  partial=upright,
]{unicode-math}

\usepackage[autostyle]{csquotes}

\usepackage[
  locale=DE,                   % deutsche Einstellungen
  separate-uncertainty=true,   % Immer Fehler mit \pm
  per-mode=symbol-or-fraction, % m/s im Text, sonst Brüche
]{siunitx}

\usepackage[version=3]{mhchem}

\usepackage{xfrac}

\usepackage[section, below]{placeins}
\usepackage[
  labelfont=bf,        % Tabelle x: Abbildung y: ist jetzt fett
  font=small,          % Schrift etwas kleiner als Dokument
  width=0.9\textwidth, % maximale Breite einer Caption schmaler
]{caption}
\usepackage{subcaption}
\usepackage{graphicx}
\usepackage{grffile}

\usepackage{float}
\floatplacement{figure}{htbp}
\floatplacement{table}{htbp}

\usepackage{booktabs}

\usepackage[
  unicode,
  pdfusetitle,    % Titel, Autoren und Datum als PDF-Attribute
  pdfcreator={},  % PDF-Attribute säubern
  pdfproducer={}, % "
]{hyperref}
\usepackage{bookmark}
\usepackage[shortcuts]{extdash}


\title{3D-Filme}
\author{Arne}

\begin{document}

\section{Ziel:}
Kurze Clips in 3D erzeugen mittels der Anaglyph-Technik

\section{Benutztes Material:} 
\begin{itemize}
	\item GoPro Hero3 White
	\item GoPro Hero3 Black
	\item Farbfilter und Anaglyph-Brillen
	\item StereoMovie Maker (SMM) (Software, Windows)
	\item Bino (Software, Mac)
\end{itemize}

\section{Vorgehen:} 
\begin{enumerate}
	\item Suche nach kompatiblen Kameraeinstellungen
	\item Optimalen Abstand der Kameraabstände zueinander finden
	\item Nötigen Abstand zum Objekt finden
\end{enumerate}

\section{Ergebnisse Software:}

\begin{tabular}{l l l}
  \toprule
  Programm & Bino & SMM\\
  \midrule 
  Positiv & Schnell & Schnitt möglich\\
          & kompatibel & Verschiebung möglich\\
          & & Interlace/Chessboard\\
  Negativ & kein Speichern & Nur für MP1 und MP2\\
          & kein Schnitt & subjektiv schlechtere Ergebnisse\\
  Abwendbar für: & Schnelle Kontrolle der Videos & Aufwendigere Clips\\
  \bottomrule
\end{tabular}

\emph{Manuelles Verschieben der Videos durch SMM nur bedingt brauchbar}


% section 3d_film (end)
\section{Ergebnisse Video:}

\begin{itemize}
	\item Unterschiede der Kameramodelle in Farbtiefe, Schärfe und bei hoher Auflösung auch im Blickwinkel 
	\item Die GoPro Hero3 Black zeigt einen störenden Versatz der Ton- und Bildspur
	\item Starke Verzeichnung an den Rändern
	\item Auflösung 720p, FPS 60
  \item Mindestabstand zum Objektiv ca. 5 Meter, da andernfalls zu starkes \enquote{Ghosting}
	\item Teilweise Verzögerung der Tonspuren
	\item Rot- und Blaufilter der 3D-Brille sind nicht mit der Farbe des Monitors kompatibel, sodass es zu "`Ghosting"' kommt
	\item Die Interlacing-Methode ist möglich, das Ghosting ist stark reduziert, jedoch die Film- und Farbqualität wesentlich schlechter
\end{itemize}

\section{Ideen:}
\begin{itemize}
	\item Rotbild zu stark, evtl. abschwächen
	\item Mindestabstand einhalten
	\item Andere Brillenfilter
	\item Andere 3D-Technik
	\item Videos mit zusätzlichem Farbfilter aufnehmen
	\item Andere Kameras ohne Weitwinkelobjektiv (bspw. identische Smartphones) verwenden
	\item Schwarz-Weiß Videos statt Farbe
	\item Bessere Bearbeitungssoftware ([Parallel-]Schnitt- und 3D-Programme)
\end{itemize}


\section{Bisherige Umsetzung der Ideen}

\begin{itemize}
\item Die Verwendung von zusätzlichen Filtern (sowohl eines Roten, wie auch eines komplementären Grünen) führte nicht zu einer Verringerung des Ghostings
\item Das Einhalten des Mindestabstands führte zum gewünschten Effekt
\item Indem zwei oder drei 3D-Brillen übereinander getragen werden, kann \enquote{Ghosting} reduziert werden, da die Farbfilter der Brillen per se zu schwach sind.
\end{itemize}


Da die GoPro-Kameras für 3D-Filme offensichtlich nicht geeignet sind, wurden testweise zwei gleiche iPhones verwendet.

\begin{itemize}
\item Verzeichnung ist kein Problem
\item Die Einzelvideos der Kameras unterschieden sich nicht merklich
\item Der Mindestabstand zu den Kameras ist deutlich geringer, sodass auch sich auf die Kamera zu bewegende Objekte gefilmt werden können
\item Ghosting tritt trotzdem auf, ist aber schwächer
\item Durch zwei 3D-Brillen, die übereinander getragen werden, können die Farben vollständig gefiltert werden, sodass kein Ghosting auftritt
\end{itemize}

\textbf{Es konnten sehr gute, bewegte 3D-Clips mithilfe der iPhones erzeugt werden.}

\end{document}
