\documentclass[
  bibliography=totoc,     % Literatur im Inhaltsverzeichnis
  captions=tableheading,  % Tabellenüberschriften
  titlepage=firstiscover, % Titelseite ist Deckblatt
]{scrartcl}

\usepackage{fixltx2e}
\usepackage[aux]{rerunfilecheck}

\usepackage{polyglossia}
\setmainlanguage{german}

\usepackage{amsmath}
\usepackage{amssymb}
\usepackage{mathtools}

\usepackage{fontspec}
\defaultfontfeatures{Ligatures=TeX}

\usepackage[
  math-style=ISO,
  bold-style=ISO,
  sans-style=italic,
  nabla=upright,
  partial=upright,
]{unicode-math}

\usepackage[autostyle]{csquotes}

\usepackage[
  locale=DE,                   % deutsche Einstellungen
  separate-uncertainty=true,   % Immer Fehler mit \pm
  per-mode=symbol-or-fraction, % m/s im Text, sonst Brüche
]{siunitx}

\usepackage[version=3]{mhchem}

\usepackage{xfrac}

\usepackage[section, below]{placeins}
\usepackage[
  labelfont=bf,        % Tabelle x: Abbildung y: ist jetzt fett
  font=small,          % Schrift etwas kleiner als Dokument
  width=0.9\textwidth, % maximale Breite einer Caption schmaler
]{caption}
\usepackage{subcaption}
\usepackage{graphicx}
\usepackage{grffile}

\usepackage{float}
\floatplacement{figure}{htbp}
\floatplacement{table}{htbp}

\usepackage{booktabs}

\usepackage[
  unicode,
  pdfusetitle,    % Titel, Autoren und Datum als PDF-Attribute
  pdfcreator={},  % PDF-Attribute säubern
  pdfproducer={}, % "
]{hyperref}
\usepackage{bookmark}
\usepackage[shortcuts]{extdash}


\title{Bau einer Wasserrakete}
\subtitle{Ein Projekt der PeP et Al. Sommerakademie 2014}
\date{24. -- 31. August 2014}
\author{
  Rober Rauter
}

\begin{document}
\maketitle
\tableofcontents

\section{Einleitung}
Eine flugfähige Wasserrakete, die den Versuch des letzten Jahres fortsetzt und höhentechnisch übertrifft, soll entwickelt werden. Dazu sollen zusätzliche Ausstattungsmerkmale, wie z.B. ein Fallschirm, verbaut werden.
Demnach soll ein wassergetriebenes Hochleistungs-Fluggerät allerneuster Generation gebaut werden (WOLFGANG).

\section{Erste Arbeitsschritte}
Eine \SI{1}{\liter}-Colaflasche wurde mit einer \SI{0.5}{\liter}-Colaflasche verbunden. Dazu wurde in den Boden der \SI{1}{\liter}-Flasche ein Loch von der Größe der Öffnung der Flasche gebohrt. Die kleinere Flasche wurde rein gedreht und mit PET-Kleber verbunden.
Beim Test der Konstruktion stellte sich heraus, dass diese erste Version der Wasserrakete bei einem Druck von 4 bar nicht dicht war, die Klebestelle stellte sich als das undichte Element heraus.

\section{Weitere Durchführung}
Der nächste Versuch wurde mit einer Kombination aus einer \SI{1.5}{\liter}- und einer \SI{1}{\liter}-Flasche umgesetzt.
Wieder wurde ein Loch in den Boden der größeren Flasche gebohrt, und die kleinere Flasche wurde mit PET-Kleber an dem Plastikring unterhalb des Schraubverschlusses beschichtet, reingedreht und mit einem Hammerschlag bis zum Plastikring in die \SI{1.5}{\liter}-Flasche hereingetrieben.
Auf das Gewinde der Flasche wurde ein Hahnanschluss aufgeschraubt, welcher auf die vorhandene Startvorrichtung passt. Die Startvorrichtung wurde vorher gekauft und ist mit einem Autoventil versehen. Dort wurde eine Fußpumpe angeschlossen, die Drücke bis zu 10 bar bereitstellen kann. 
Nachdem die Naht der Rakete (Flasche-Flasche-Übergang) getrocknet war, wurde eine zweite Schicht Kleber aufgetragen, damit die Rakete auch bei höheren Drücken wasser- und luftdicht  bleibt.
  
\section{Fallschirmkonstruktion}
Es wurde ein Fallschirm konstruiert. Dazu wurde ein Achteck aus einem Müllsack ausgeschnitten, die Ecken wurden mit Spiegelpapier verstärkt, sodass diese die Aufgaben von Ösen übernahmen. Nylonfäden, die an allen Ösen des Fallschirms gebunden wurden, wurden an deren Ende mit Tape zusammengeklebt.
Der Fallschirm stellte sich als funktionsfähig heraus (wurde lose oben auf die Spitze der Rakete gelegt), der Einsatz selbigen verringerte die Flughöhe jedoch erheblich. Er öffnete sich dennoch erwartungsgemäß und sorgte für eine sehr ruhige Landung.

\section{Leitwerk}
Für diese Rakete wurde ein rundes Leitwerk erstellt. Dazu wurde ein Ring aus einer Haribodose geschnitten, mit Schaschlikspießen verbunden und durch an der Rakete angebrachte Strohhalme befestigt. Dies hatte zur Folge, dass das Leitwerk sich beim Start selbstständig ausfährt. Beim Einsatz stellte sich heraus, dass diese neue Konstruktion des Leitwerks eine große Verbesserung der erreichten Höhe zur Folge hatte.
\section{Messergebnisse}
Beim Testen der Rakete bei verschiedenen Befüllungsvolumina von Wasser stellte sich bei einem Druck von 4,8 bar eine optimale Wassermenge von 750 ml heraus, bei einer \SI{2.5}{\liter}-Flasche.

\section{Probleme und Verbesserungen}
\begin{itemize}
	\item Die Rakete weist an den Klebestellen geringe Druckbeständigkeit auf, nur PET-Kleber verbindet die Flasche wirklich effizient. Möglicherweise muss über alternative Abdichtungsmöglichkeiten nachgedacht werden.
	\item Der Fallschirm entfaltet sich nur, wenn die Rakete am höchsten Punkt senkrecht nach oben zeigt. Eine bessere Unterbringung und Falttechnik ist anzustreben.
	\item Die Abschussbasis startet die Rakete automatisch und unkontrolliert bei 5 bar.
	\item Die Höhenmessung über eine markierte Schnur beeinträchtigt zum einen die Fallhöhe, und ist zum anderen recht ungenau. 
\end{itemize}
\end{document}
