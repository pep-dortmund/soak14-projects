\documentclass[
  bibliography=totoc,     % Literatur im Inhaltsverzeichnis
  captions=tableheading,  % Tabellenüberschriften
  titlepage=firstiscover, % Titelseite ist Deckblatt
]{scrartcl}

\usepackage{fixltx2e}
\usepackage[aux]{rerunfilecheck}

\usepackage{polyglossia}
\setmainlanguage{german}

\usepackage{amsmath}
\usepackage{amssymb}
\usepackage{mathtools}

\usepackage{fontspec}
\defaultfontfeatures{Ligatures=TeX}

\usepackage[
  math-style=ISO,
  bold-style=ISO,
  sans-style=italic,
  nabla=upright,
  partial=upright,
]{unicode-math}

\usepackage[autostyle]{csquotes}

\usepackage[
  locale=DE,                   % deutsche Einstellungen
  separate-uncertainty=true,   % Immer Fehler mit \pm
  per-mode=symbol-or-fraction, % m/s im Text, sonst Brüche
]{siunitx}

\usepackage[version=3]{mhchem}

\usepackage{xfrac}

\usepackage[section, below]{placeins}
\usepackage[
  labelfont=bf,        % Tabelle x: Abbildung y: ist jetzt fett
  font=small,          % Schrift etwas kleiner als Dokument
  width=0.9\textwidth, % maximale Breite einer Caption schmaler
]{caption}
\usepackage{subcaption}
\usepackage{graphicx}
\usepackage{grffile}

\usepackage{float}
\floatplacement{figure}{htbp}
\floatplacement{table}{htbp}

\usepackage{booktabs}

\usepackage[
  unicode,
  pdfusetitle,    % Titel, Autoren und Datum als PDF-Attribute
  pdfcreator={},  % PDF-Attribute säubern
  pdfproducer={}, % "
]{hyperref}
\usepackage{bookmark}
\usepackage[shortcuts]{extdash}


\title{Weiterentwicklung eines Solarofens}
\subtitle{Ein Projekt der PeP et Al. Sommerakademie 2014}
\date{24. -- 31. August 2014}

\author{

}


\begin{document}
\maketitle
\tableofcontents

Aufbau

Als Ofen wird eine kleinere Bambuskiste benutzt. Diese sitzt in einer größeren Holzkiste, sodass eine Luftschicht zwischen den Kisten existiert. Diese Luftschicht kann durch verschiedene Dämmmaterialien ausgetauscht werden. Ein Glasdeckel verschließt die innere Kiste luftdicht aber lichtdurchlässig. Die innere Holzkiste wird vollständig mit Aluminiumplatten ausgekleidet und diese dann mit schwarzem Termolack bemalt. Sonnenlicht wird mit einem Spiegel am Deckel ins Innere der Kiste reflektiert. 

Durchführung

Der Ofen wird Richtung Sonne ausgerichtet und der Deckel in einen Winkel gestellt, der das Sonnenlicht ins Innere reflektiert. Wird ein Körper im Inneren platziert, so kann er erwärmt werden. 
Der Deckel wird durch eine Seilkonstruktion im richtigen Winkel gehalten und sollte alle Zehn Minuten nachjustiert werden.

Beobachtung

Nach einer Stunde wird eine Erwärmung des mit reingelegten Thermometers festgestellt. 

Auswertung

Die in die Kiste reflektierten Sonnenstrahlen verlieren beim Durchdringen des Glasdeckels ihre UV-Strahlung, da diese reflektiert wird. Das sichtbare Licht erwärmt die schwarzen Wände der inneren Kiste, welche dann ihrerseits thermische Strahlung emittieren. Diese transmittiert nicht durch die Glasplatte und bleibt im Ofen.
\end{document}
