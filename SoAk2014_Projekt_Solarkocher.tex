\documentclass[
  bibliography=totoc,     % Literatur im Inhaltsverzeichnis
  captions=tableheading,  % Tabellenüberschriften
  titlepage=firstiscover, % Titelseite ist Deckblatt
]{scrartcl}

\usepackage{fixltx2e}
\usepackage[aux]{rerunfilecheck}

\usepackage{polyglossia}
\setmainlanguage{german}

\usepackage{amsmath}
\usepackage{amssymb}
\usepackage{mathtools}

\usepackage{fontspec}
\defaultfontfeatures{Ligatures=TeX}

\usepackage[
  math-style=ISO,
  bold-style=ISO,
  sans-style=italic,
  nabla=upright,
  partial=upright,
]{unicode-math}

\usepackage[autostyle]{csquotes}

\usepackage[
  locale=DE,                   % deutsche Einstellungen
  separate-uncertainty=true,   % Immer Fehler mit \pm
  per-mode=symbol-or-fraction, % m/s im Text, sonst Brüche
]{siunitx}

\usepackage[version=3]{mhchem}

\usepackage{xfrac}

\usepackage[section, below]{placeins}
\usepackage[
  labelfont=bf,        % Tabelle x: Abbildung y: ist jetzt fett
  font=small,          % Schrift etwas kleiner als Dokument
  width=0.9\textwidth, % maximale Breite einer Caption schmaler
]{caption}
\usepackage{subcaption}
\usepackage{graphicx}
\usepackage{grffile}

\usepackage{float}
\floatplacement{figure}{htbp}
\floatplacement{table}{htbp}

\usepackage{booktabs}

\usepackage[
  unicode,
  pdfusetitle,    % Titel, Autoren und Datum als PDF-Attribute
  pdfcreator={},  % PDF-Attribute säubern
  pdfproducer={}, % "
]{hyperref}
\usepackage{bookmark}
\usepackage[shortcuts]{extdash}


\title{Aufbau eines Solarkochers}
\subtitle{Ein Projekt der PeP et Al. Sommerakademie 2014}
\date{24. -- 31. August 2014}

\author{
	Philipp Hoffmann \texorpdfstring{\and}{,}
	Fabian Götz}

\begin{document}
\maketitle
\tableofcontents


\section{Aufbau}

Eine mit Spiegelfolie überzogende Satellitenschüssel wird mit einer Halterung auf einen Holzständer montiert.
Der Holzständer ist ein auf einem Kreuz montiertes Brett, welches die richtigen Maße für die Halterung der Satellitenschüssel hat.
Diese ist in ihrer Neigung verstellbar.
Als Halterung für das zu kochende Objekt wurde ein Dreibein zusammengebaut.
Die Dose oder der Topf hängen von einer Seilkonstruktion an dem Dreibein und sollten für das Experiment spezialisiert werden.
Dies ist durch schwarze Ofenfarbe realisierbar, wodurch die Wärmestrahlung besser absorbiert wird. 


\section{Durchführung}

Das zu kochende Objekt wird von der Seilkonstruktion in den Fokuspunkt der Satellitenschüssel gehängt. Der Fokuspunkt kann durch eine weiße Pappe, die von der Satellitenschüssel weg bewegt wird, herausgefunden werden. Nach wenigen Sekunden können erste Anzeichen einer Erwärmung beobachtet werden.


\section{Beobachtung}

Die schwarz angemalte Dose hat nach einigen Minuten angefangen zu kochen. Wurde die Lichteinstrahlung durch Wolken vor der Sonne oder abdecken der Schüssel reduziert, hörte der Kochvorgang instantan auf. Mit erneuter Einstrahlung fing das Wasser wieder zu kochen an. 
Ein weiteres Experiment wurde mit einem Holzbrett im Brennpunkt durchgeführt. Nach nur wenigen Sekunden war eine schwarze Verfärbung und Qualm zu beobachten. Kurz danach brannte das Brett.



\section{Auswertung}

Aus dem Teil des Experimentes, in dem Wasser gekocht wurde lässt sich schließen, dass die Strahlungsleistung des Solarkochers nur bei voller Sonneneinstrahlung ausreicht um Wasser zum Sieden zu bringen.
Da das Holz bei voller Sonneneinstrahlung angefangen hat zu brennen, kann auf eine Temperatur bis zu \SI{300}{\celsius} im Brennpunkt geschlossen werden.
\end{document}
